\documentclass[11pt]{article}

% -------------- PREÁMBULO ---------------

\usepackage[spanish]{babel} 
\usepackage[utf8]{inputenc} 
\usepackage{amsmath}
\usepackage{amssymb} 
\usepackage{graphicx}           % Incluir imágenes en LaTeX
\usepackage{color}              % Para colorear texto
\usepackage{subfigure}          % subfiguras
\usepackage{float}              % Podemos usar el especificador [H] en las figuras para que se queden donde queramos
\usepackage{capt-of}            % Permite usar etiquetas fuera de elementos flotantes (etiquetas de figuras)
\usepackage{sidecap}            % Para poner el texto de las imágenes al lado
\sidecaptionvpos{figure}{c}     % Para que el texto se alinee al centro vertical
\usepackage{caption}            % Para poder quitar numeracion de figuras
\usepackage{anysize}            % Para personalizar el ancho de  los márgenes
\marginsize{2cm}{2cm}{2cm}{2cm} % Izquierda, derecha, arriba, abajo
\usepackage{multicol}
\usepackage{multirow}
\setlength{\columnsep}{1cm}

% Para agregar encabezado y pie de página
\usepackage{fancyhdr} 
\usepackage{clrscode3e}

\usepackage{listings}
\pagestyle{fancy}
\fancyhf{}
\fancyhead[L]{\footnotesize USB}                  % Encabezado izquierda
\fancyhead[R]{\footnotesize Estructuras de datos en MIPS}  
\fancyfoot[R]{\footnotesize Informe}              % Pie derecha
\fancyfoot[C]{\thepage}                           % centro
\fancyfoot[L]{\footnotesize Ingeniería de la Computación}  %izquierda
\renewcommand{\footrulewidth}{0.4pt}

\newcommand{\coment}[1]{}
\definecolor{BurntOrange}{RGB}{247,148,42}

\begin{document}

% ----------------- PORTADA -----------------

\begin{center} 
   \newcommand{\HRule}{\rule{\linewidth}{0.5mm}}  

   \begin{minipage}{0.48\textwidth}
      \begin{center}
         \includegraphics[scale = 0.5]{logo.png}
      \end{center}
   \end{minipage}

   \vspace*{1.0cm}                       
   \textsc{\huge Universidad Simón Bolívar} \\ [1.5cm] 

   \begin{minipage}{0.9\textwidth} 
      \begin{center}                                                             
         \textsc{\LARGE Informe de Proyecto I }
      \end{center}
   \end{minipage} \\ [3cm]

   \vspace*{1cm}                                                                              
   \HRule \\ [0.4cm]                                                  
   {\huge \bfseries Estructuras de Datos en MIPS} \\ [0.4cm] 
   \HRule \\ [4cm]

   \begin{minipage}{\textwidth} 
      \begin{flushleft} \large    
         \textbf{\underline{Autor:}} \\ 
         Christopher Gómez (18-10892)\\
         Ka Fung (18-10492)\\
      \end{flushleft}
   \end{minipage}

   \begin{minipage}{\textwidth}    
      \vspace{-0.6cm}  
      \begin{flushright} \large    
         \textbf{\underline{Profesor:}} \\  
         Eduardo Blanco  
      \end{flushright}        
   \end{minipage} 

   \vspace*{1cm}
   \flushleft{\textbf{\Large Organización del Computador (CI3815)} }\\
   \vspace{2cm}  

   \begin{center} 
      {\large \today} 
   \end{center}     
\end{center}                                                      
                                                               
\newpage
                                                    
% -------------------------------------------

\section{Pseudocódigo}

Se presentan a continuación los pseudocódigos del programa que
se desea implementar. \\

Primeramente, se necesita extraer los datos necesarios de cada
archivo de entrada, para ello, se usan tablas de hash y listas.

\begin{codebox}
   \Procname{$\proc{Extraer-Datos}$}
   \li \id{archivoEst} = Leer archivo de estudiantes
   \li \id{tablaHashEst} = \kw{new} \proc{TablaHash}(101)

   \li \For linea in archivoEst:
   \li \Do
      \id{carnet} = Guardar carnet
      \li \id{nombre} = Guardar nombre
      \li \id{indice} = Guardar índice
      \li \id{creditosAprob} = Guardar número de créditos aprobados
      \li \id{est} = \kw{new} \proc{Estudiante}(\id{carnet}, \id{nombre}, \id{indice}, \id{creditosAprob})
      \li \id{tablaHashEst}.\proc{Insertar}(\id{carnet}, \id{est})
      \End

   \li
   \li \id{archivoMat} = Leer archivo de materias
   \li \id{tablaHashMat} = \kw{new} \proc{TablaHash}(101)
   \li \id{listaMat} = \kw{new} \proc{Lista}()
   \li \For linea in archivoMat:
   \li \Do
      \id{codigo} = Guardar codigo
      \li \id{nombre} = Guardar nombre
      \li \id{creditos} = Guardar creditos
      \li \id{numCupos} = Guardar número de cupos
      \li \id{minCreditos} = Guardar mínimo de créditos
      \li \id{mat} = \kw{new} \proc{Materia}(\id{codigo}, \id{nombre}, \id{creditos}, \id{numCupos}, \id{minCreditos})
      \li \id{tablaHashMat}.\proc{Insertar}(\id{carnet}, \id{est})
      \li \id{listaMat}.\proc{Insertar-Ordenado}(\id{codigo}, \id{f})
      \End

   \li
   \li \id{listaSol} = \kw{new} \proc{Lista}()
   \li \id{archivoSol} = Leer archivo de solicitudes
   \li \For linea in archivoSol:
   \li \Do
      \id{carnet} = Guardar carnet del estudiante
      \li \id{est} = \id{tablaHashEst}.\proc{Obtener-Valor}(\id{carnet})
      \li \id{codigo} = Guardar codigo de la materia
      \li \id{mat} = \id{tablaHashMat}.\proc{Obtener-Valor}(\id{codigo})
      \li \id{sol} = \kw{new} \proc{Solicitud}(\id{est}, \id{mat}, \id{op})
      \li \id{listaSol}.\proc{Insertar}(\id{sol}) \label{li:Extraer-Datos-final}
      \End
   \End
   \end{codebox}

   Al terminar este pseudocódigo, se debe tener una lista de solicitudes,
   una lista de códigos de materias en orden lexicográfico, una tabla de
   estudiantes, y una tabla de materias, la idea ahora es procesar la lista
   de solicitudes para que cada materia tenga una lista de estudiantes inscritos.

   \begin{codebox}
      \setlinenumberplus{li:Extraer-Datos-final}{1}
      \Procname{$\proc{Procesar-Solicitudes}$}
      \li \For sol in listaSol:
      \Do
         \li \id{est} = \attrib{sol}{estudiante}
         \li \id{mat} = \attrib{sol}{materia}
         \li \id{mat}.\proc{Agregar-Estudiante}(\id{est}) \label{li:Procesar-Solicitudes-final}
         \End
      \End
   \end{codebox}

   Ahora, cada materia contiene una lista con los estudiantes inscritos. Se
   supone que la estructura se encarga de mantener actualizado el número de
   cupos y de agregar en orden a los estudiantes en su lista de estudiantes.
   Así, para finalizar esta primera etapa solamente resta escribir en el
   archivo de salida cada materia con sus estudiantes inscritos.

   \begin{codebox}
      \setlinenumberplus{li:Procesar-Solicitudes-final}{1}
      \Procname{$\proc{Generar-Archivo-Tentativo}$}
      \li \id{archivoTen} = Abrir archivo tentativo a escribir
      \li \For mat in listaMat:
      \Do
         \li \id{archivoTen}.\proc{Escribir}(`$<$\attrib{mat}{codigo}$>$ ')
         \li \id{archivoTen}.\proc{Escribir}(```$<$\attrib{mat}{nombre}$>$'' ')
         \li \id{archivoTen}.\proc{Escribir}(`$<$\attrib{mat}{numCupos}$>\backslash$n')
         
         \li \For est in \attrib{mat}{estudiantes}:
         \Do
            \li \id{archivoTen}.\proc{Escribir}(`   $<$\attrib{est}{carnet}$>$ ')
            \li \id{archivoTen}.\proc{Escribir}(`$<$\attrib{est}{nombre}$>\backslash$n') \label{li:Generar-Archivo-Tentativo-final}
            \End
         \End
         \End
      \End
   \end{codebox}

   Luego, se procesa las solicitudes de corrección. Al terminar el siguiente pseudocódigo, 
   se debe tener una lista de solicitudes de corrección. Se procesa la lista de solicitudes 
   para que cada materia tenga una lista de estudiantes inscritos y eliminados. Se ordena 
   la lista de estudiantes de cada materia según la prioridad: M1.5 Se le da prioridad a los 
   estudiantes con menor número de créditos aprobados. 

   \begin{codebox}
      \Procname{$\proc{Extraer-Datos-Correccion}$}
      \li
      \li \id{listaSolCor} = \kw{new} \proc{Lista}()
      \li \id{archivoCor} = Leer archivo de solicitudes de corrección
      \li \For linea in archivoCor:
      \li \Do
         \id{carnet} = Guardar carnet del estudiante
         \li \id{est} = \id{tablaHashEst}.\proc{Obtener-Valor}(\id{carnet})
         \li \id{codigo} = Guardar codigo de la materia
         \li \id{mat} = \id{tablaHashMat}.\proc{Obtener-Valor}(\id{codigo})
         \li \id{op} = Guardar operación de la solicitud
         \li \id{sol} = \kw{new} \proc{Solicitud}(\id{est}, \id{mat}, \id{op})
         \li \id{listaCor}.\proc{Insertar}(\id{sol}) \label{li:Extraer-Datos-Correccion-final}
         \End
      \End
   \end{codebox}

   \begin{codebox}
      \setlinenumberplus{li:Extraer-Datos-final}{1}
      \Procname{$\proc{Procesar-Solicitudes-Correccion}$}
      \li \For sol in listaCor:
      \Do
         \li \id{est} = \attrib{sol}{estudiante}
         \li \id{mat} = \attrib{sol}{materia}

         \li \If \attrib{sol}{op} == 'E':
         \Do 
            \li \id{mat}.\proc{Eliminar-Estudiante}(\id{est})
            \li \attrib{mat}{cupos}++
         \li \Else:
            \li \id{mat}.\proc{Agregar-Estudiante}(\id{est})
            \li \attrib{mat}{cupos}- -
         \End

         

         \label{li:Procesar-Solicitudes-Correccion-final}
         \End
      \End
   \end{codebox}

   Finalmente, se escribe en el archivo de salida cada materia con sus 
   estudiantes inscritos y eliminados.

   \begin{codebox}
      \setlinenumberplus{li:Procesar-Solicitudes-final}{1}
      \Procname{$\proc{Generar-Archivo-Definitivo}$}
      \li \id{archivoDef} = Abrir archivo definitivo a escribir
      \li \For mat in listaMat:
      \Do
         \li \id{archivoDef}.\proc{Escribir}(`$<$\attrib{mat}{codigo}$>$ ')
         \li \id{archivoDef}.\proc{Escribir}(```$<$\attrib{mat}{nombre}$>$'' ')
         \li \id{archivoDef}.\proc{Escribir}(`$<$\attrib{mat}{numCupos}$>\backslash$n')
         
         \li \For est in \attrib{mat}{estudiantes}:
         \Do
            \li \id{archivoTen}.\proc{Escribir}(`$<$\attrib{est}{carnet}$>$ ')
            \li \id{archivoTen}.\proc{Escribir}(`$<$\attrib{est}{nombre}$>$ ') 
            \li \id{archivoTen}.\proc{Escribir}(`$<$\attrib{est}{op}$>\backslash$n') 
            
            \label{li:Generar-Archivo-Tentativo-final}
            \End
         \End
         \End
      \End
   \end{codebox}

   \section{Estructuras utilizadas}

   En la seccion anterior se menciona el uso de distintas estructuras de datos
   utilizadas en el diseño del programa. En esta sección se describe cada una
   de ellas, junto con sus atributos y operaciones.

   \begin{itemize}
      \item \proc{Par}:
      \begin{itemize}
         \item Atributos:

         \begin{itemize}
            \item Primer elemento.
            \item Segundo elemento.
         \end{itemize}
      \end{itemize}

      \begin{itemize}
         \item Operaciones:

         \begin{itemize}
            \item \proc{Crear}(\id{primero}, \id{segundo})
         \end{itemize}
      \end{itemize}

      \item \proc{Lista}:

      \begin{itemize}
         \item Atributos:

         \begin{itemize}
            \item Cabeza.
            \item Tamaño.
         \end{itemize}
      \end{itemize}

      \begin{itemize}
         \item Operaciones:

         \begin{itemize}
            \item \proc{Crear}()
            \item \proc{Insertar}(\id{elemento})
            \item \proc{Insertar-Ordenado}(\id{elemento}, \id{f})
         \end{itemize}
      \end{itemize}

      \item \proc{TablaHash}:
      
      \begin{itemize}
         \item Atributos:

         \begin{itemize}
            \item Tamaño.
            \item Tabla.
         \end{itemize}
      \end{itemize}

      \begin{itemize}
         \item Operaciones:
         
         \begin{itemize}
            \item \proc{Crear}(\id{tam})
            \item \proc{Insertar}(\id{clave}, \id{valor})
            \item \proc{Obtener-Valor}(\id{clave})
         \end{itemize}
      \end{itemize}

      \item \proc{Estudiante}:
      \begin{itemize}
         \item Atributos:

         \begin{itemize}
            \item Carnet.
            \item Nombre.
            \item Índice.
            \item Créditos aprobados.
         \end{itemize}
      \end{itemize}

      \begin{itemize}
         \item Operaciones:
         
         \begin{itemize}
            \item \proc{Crear}(\id{carnet}, \id{nombre}, \id{indice}, \id{creditosAprob})
         \end{itemize}
      \end{itemize}

      \item \proc{Materia}:
      \begin{itemize}
         \item Atributos:

         \begin{itemize}
            \item Código.
            \item Nombre.
            \item Número de créditos.
            \item Cupos.
            \item Número mínimo de créditos aprobados.
         \end{itemize}
      \end{itemize}

      \begin{itemize}
         \item Operaciones:
         
         \begin{itemize}
            \item \proc{Crear}(\id{codigo}, \id{nombre}, \id{creditos}, \id{numCupos}, \id{minCreditos})
            \item \proc{Aumentar-Cupo}()
            \item \proc{Disminuir-Cupo}()
            \item \proc{Agregar-Estudiante}(\id{Estudiante})
            \item \proc{Eliminar-Estudiante}(\id{Estudiante})
         \end{itemize}
      \end{itemize}


      \item \proc{Solicitud}:
      \begin{itemize}
         \item Atributos:

         \begin{itemize}
            \item Estudiante.
            \item Materia.
            \item Operación.
         \end{itemize}
      \end{itemize}

      \begin{itemize}
         \item Operaciones:
         
         \begin{itemize}
            \item \proc{Crear}(\id{est}, \id{mat}, \id{op})
         \end{itemize}
      \end{itemize}
      
   \end{itemize}



\end{document}