\documentclass[11pt]{article}

% -------------- PREÁMBULO ---------------

\usepackage[spanish]{babel} 
\usepackage[utf8]{inputenc} 
\usepackage{amsmath}
\usepackage{amssymb} 
\usepackage{graphicx}           % Incluir imágenes en LaTeX
\usepackage{color}              % Para colorear texto
\usepackage{subfigure}          % subfiguras
\usepackage{float}              % Podemos usar el especificador [H] en las figuras para que se queden donde queramos
\usepackage{capt-of}            % Permite usar etiquetas fuera de elementos flotantes (etiquetas de figuras)
\usepackage{sidecap}            % Para poner el texto de las imágenes al lado
\sidecaptionvpos{figure}{c}     % Para que el texto se alinee al centro vertical
\usepackage{caption}            % Para poder quitar numeracion de figuras
\usepackage{anysize}            % Para personalizar el ancho de  los márgenes
\marginsize{2cm}{2cm}{2cm}{2cm} % Izquierda, derecha, arriba, abajo
\usepackage{multicol}
\usepackage{multirow}
\setlength{\columnsep}{1cm}

% Para agregar encabezado y pie de página
\usepackage{fancyhdr} 
\usepackage{clrscode3e}

\usepackage{listings}
\pagestyle{fancy}
\fancyhf{}
\fancyhead[L]{\footnotesize USB}                  % Encabezado izquierda
\fancyhead[R]{\footnotesize Estructuras de datos en MIPS}  
\fancyfoot[R]{\footnotesize Informe}              % Pie derecha
\fancyfoot[C]{\thepage}                           % centro
\fancyfoot[L]{\footnotesize Ingeniería de la Computación}  %izquierda
\renewcommand{\footrulewidth}{0.4pt}

\newcommand{\coment}[1]{}
\definecolor{BurntOrange}{RGB}{247,148,42}

\begin{document}

% ----------------- PORTADA -----------------

\begin{center} 
   \newcommand{\HRule}{\rule{\linewidth}{0.5mm}}  

   \begin{minipage}{0.48\textwidth}
      \begin{center}
         \includegraphics[scale = 0.5]{logo.png}
      \end{center}
   \end{minipage}

   \vspace*{1.0cm}                       
   \textsc{\huge Universidad Simón \\ \vspace{5px} Bolívar} \\ [1.5cm] 

   \begin{minipage}{0.9\textwidth} 
      \begin{center}                                                             
         \textsc{\LARGE Informe de Proyecto I }
      \end{center}
   \end{minipage} \\ [3cm]

   \vspace*{1cm}                                                                              
   \HRule \\ [0.4cm]                                                  
   {\huge \bfseries Estructuras de Datos en MIPS} \\ [0.4cm] 
   \HRule \\ [4cm]

   \begin{minipage}{\textwidth} 
      \begin{flushleft} \large    
         \textbf{\underline{Autor:}} \\ 
         Christopher Gómez \\
         Ka Fung \\
      \end{flushleft}
   \end{minipage}

   \begin{minipage}{\textwidth}    
      \vspace{-0.6cm}  
      \begin{flushright} \large    
         \textbf{\underline{Profesor:}} \\  
         Eduardo Blanco  
      \end{flushright}        
   \end{minipage} 

   \vspace*{1cm}
   \flushleft{\textbf{\Large Organización del Computador (CI3815)} }\\
   \vspace{2cm}  

   \begin{center} 
      {\large \today} 
   \end{center}     
\end{center}                                                      
                                                               
\newpage
                                                    
% -------------------------------------------

\section{Pseudocódigo}

\begin{codebox}
   \Procname{$\proc{Main}$}
   \li \id{archivoEst} = Leer archivo de estudiantes
   \li \id{tablaHashEst} = \textbf{new} \proc{TablaHash}(101)

   \li \For linea in archivoEst:
   \li \Do
      \id{carnet} = Guardar carnet
      \li \id{nombre} = Guardar nombre
      \li \id{indice} = Guardar índice
      \li \id{creditosAprob} = Guardar número de créditos aprobados
      \li \id{est} = \textbf{new} \proc{Estudiante}(carnet, nombre, indice, creditosAprob)
   \End
   \End
   \end{codebox}

\end{document}